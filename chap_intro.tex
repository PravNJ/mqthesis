\begin{savequote}[45mm]
The new star in Cygnus that I first observed on August 8, 1600, was initially of third magnitude. I determined its position by measuring its distance from Vega and Albireo. It remains in this position but now is no brighter than 5th magnitude
\qauthor{Willem Janszoon Blaeu}
\end{savequote}

\chapter{P Cygni Stars}

\section{Background}

P Cygni (or 34 Cygni) is a luminous blue variable star (LBV) that has been studied extensively \cite{1953PDAO....9....1B, hutchings1969expanding, elliott20225, underhill1966supergiants}. Willem Janszoon Blaeu, a Dutch cartographer and student of the astronomer Tycho Brahe is considered to have provided the first known set of observations of 34 Cygni in the year 1600 \cite{deGrootPCygni}. The stellar spectrum of 34 Cygni is peculiar. It exhibits the characteristics of a B type supergiant except that almost all absorption lines are blue shifted with a red shifted emission component \cite{hutchings1969expanding}. 

\includegraphics{pcygniprofile}