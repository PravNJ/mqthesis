\chapter{Abstract}

The advent of massive stellar spectroscopic surveys with hundreds of thousands or even millions of spectra presents serious challenges for the identification and classification of relatively atypical objects, such as emission-line stars. To date, a variety of machine-learning techniques has been applied to the identification of emission-line stars, but in most cases actual classification has been carried out by humans, even for datasets comprising tens of thousands of spectra. As spectroscopic surveys grow even larger – by orders of magnitude – such solutions will become unfeasible.

This thesis seeks to address the twin problems of identification and classification for large spectroscopic datasets through the application of machine-learning methods to spectra from the GALAH survey. The GALAH survey is a million-star high-resolution spectroscopic survey of the Milky Way which uses the HERMES spectrograph at the Anglo Australian Telescope. The most recent public data release from GALAH, its third (DR3), contains more than 600,000 high-resolution spectra. Such a large volume of data necessitates the use of semi-automated and automated data analytic pipelines over manual methods. This work presents a novel data-driven approach to identify and classify emission-line stars in the GALAH database, utilising an unsupervised machine learning method that performs spectral morphology based clustering, with a particular focus on stars showing P Cygni and inverse P Cygni Hα line profiles. Such approaches can potentially be used for finding rare or unusual objects in even larger datasets, based on their detailed spectral properties.

